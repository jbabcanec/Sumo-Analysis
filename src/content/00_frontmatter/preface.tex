\chapter*{Preface: How to Read This Book}
\addcontentsline{toc}{chapter}{Preface}

\section*{A Personal Note}

My interest in sumo began in early 2025, though my fascination with Japanese culture extends back years. I'd studied Japanese orthography, practiced kendo briefly, absorbed the language and aesthetics of a culture that seemed to have mastered something I found compelling: the ability to codify chaos into defined, almost artistic practice. Japan takes the messiness of human activity—whether it's brewing tea, arranging flowers, or two massive men trying to throw each other out of a ring—and transforms it into something with clear rules, deep tradition, and profound meaning.

I remember reading on Japanese Wikipedia about Hoshoryu's promotion to yokozuna and wanting to understand more. The stable system intrigued me—young men living communally under brutal hierarchical structures, dedicating their lives to mastering a sport most would never succeed in. The evolution of sumo as a cultural landmark, surviving wars and modernization while maintaining rituals over a millennium old, felt like a case study in how tradition and change negotiate with each other.

But what really hooked me was the simplicity. You can understand sumo in seconds just by watching one match. The rules are clear: force your opponent out of the circle or make any part of their body besides their feet touch the ground. That's it. Yet within that simplicity exists extraordinary complexity—technique, strategy, psychology, physical conditioning. This combination of accessible surface and deep structure is characteristic of what I admire about Japanese cultural forms.

I began this book because I'm interested in statistical ramifications. American baseball recruiting has become a game of which team employs the better mathematician. Basketball analysts can predict success using hand size or seemingly arbitrary physical measurements. I wanted to explore whether similar approaches could work for sumo, and what unique statistical phenomena might emerge from a sport with such different institutional structures than Western professional athletics.

I should be transparent about methodology: I use AI extensively in writing this book. This reflects a practical division of labor based on my actual interests and skills. I design and perform all statistical analysis myself—the models, the data processing, the interpretation of results. That's what I care about and what I'm trained to do. For contextual material like historical background, I guide the subject matter and structure, but I'm not precious about writing my own version of historical facts that have been documented elsewhere. AI handles synthesis of that existing knowledge efficiently, freeing me to focus on original analytical work. The history chapter you're reading? AI-generated from primary and secondary sources, under my direction. The statistical models in later chapters? Those are mine, along with the interpretations and implications.

This isn't laziness; it's honest allocation of effort toward what I can contribute uniquely. I'm not a historian of Japanese culture or an expert on Shinto ritual. But I can build regression models, evaluate predictive accuracy, and think carefully about causal inference in observational data. That's where this book's value lies, and that's where I've invested my time.

\section*{Research Questions}

Sumo wrestling, Japan's ancient national sport, stands at a fascinating intersection of tradition and evolution. While the rituals and cultural significance remain unchanged across centuries, the athletes, techniques, and competitive dynamics have undergone remarkable transformations. This book asks fundamental questions about that evolution: What factors truly influence success in the ring? Can we identify patterns that predict future performance? How do training practices, body composition, and technique selection evolve over time in response to competitive pressures?

Modern American sports analytics has revolutionized how we understand athletic performance---from baseball's sabermetrics to basketball's spatial tracking, teams now leverage sophisticated statistical models to evaluate talent, predict outcomes, and optimize strategies. A central question driving this work is whether these same analytical frameworks can illuminate sumo wrestling, a sport with fundamentally different structure, constraints, and cultural context. Can we build rating systems that capture wrestler quality beyond win-loss records? Can injury risk be predicted and mitigated? What recruiting characteristics signal future yokozuna potential?

This investigation reveals both universal patterns that transcend sporting contexts and unique dynamics specific to sumo's institutional structure. The same statistical tools that evaluate NBA draft prospects can identify promising young rikishi, yet sumo's stable system, foreign-born wrestler caps, and rigid ranking hierarchy create analytical challenges absent from Western professional sports.

\section*{Purpose and Scope}

This book represents a comprehensive quantitative analysis of professional sumo wrestling, leveraging modern data science techniques to understand the sport's competitive dynamics, strategic elements, and physical demands. Our approach combines rigorous statistical modeling with domain expertise to illuminate patterns that have shaped sumo from 1958 to the present day.

The scope encompasses:
\begin{itemize}
\item Systematic analysis of bout outcomes using mixed-effects models
\item Technique ecology and strategic diversity across eras
\item Physical anthropometry and its relationship to success
\item Injury patterns and career longevity
\item Rating systems and predictive modeling
\item Natural experiments from policy changes
\end{itemize}

\section*{Data Sources and Reproducibility}

All analyses in this book are built on publicly available data sources, primarily:

\begin{itemize}
\item \textbf{Sumo-API}: Primary structured data source (1958--present) providing rikishi profiles, tournament results, banzuke rankings, and bout outcomes
\item \textbf{Japan Sumo Association (JSA)}: Official records for verification and supplementary information
\item \textbf{SumoDB}: Historical cross-validation and extended queries
\item \textbf{Medical Literature}: PubMed and PMC databases for injury prevalence and health outcomes
\end{itemize}

Complete code for data acquisition, processing, and analysis is available at our GitHub repository. All random seeds are fixed for reproducibility, and data snapshots are versioned to ensure consistency across analyses.

\section*{What ``Causality'' Does Not Mean Here}

This book operates primarily in the realm of observational data analysis. When we discuss relationships between variables---such as the association between technique diversity and win rates---we are describing correlations and predictive patterns, not necessarily causal mechanisms.

True causal inference requires either:
\begin{itemize}
\item Randomized controlled experiments (impossible in professional sports)
\item Natural experiments with credible identification strategies
\item Strong theoretical assumptions that may not hold
\end{itemize}

We employ quasi-experimental methods where possible, particularly around policy changes like:
\begin{itemize}
\item The 2003 abolition of the \japterm{kōshō}{kosho} system (injury leave)
\item The 2010 clarification of foreign-born wrestler quotas
\item Changes in tournament frequency and structure
\end{itemize}

However, readers should interpret findings as associations and patterns rather than definitive causal relationships unless explicitly noted otherwise.

\section*{Structural Constraints Shaping the Data}

Several institutional features of professional sumo create unique data-generating processes:

\paragraph{Foreign-Born Wrestler Cap} Each stable (\japterm{heya}{heya}) may have only one foreign-born wrestler at a time. This constraint affects recruitment patterns, career trajectories, and competitive dynamics in ways that must be accounted for in our models.

\paragraph{Same-Stable Non-Competition Rule} Wrestlers from the same stable do not face each other in regular tournament bouts (exceptions exist for playoffs). This creates systematic missing data patterns that affect rating calculations and head-to-head analyses.

\paragraph{Promotion/Demotion System} The rigid ranking system creates discrete jumps in competition level and incentive structures, particularly around the 7-7 record threshold for maintaining rank.

\paragraph{Tournament Structure Evolution} The shift to six tournaments per year in 1958 and other structural changes create era effects that must be controlled for in longitudinal analyses.

\section*{How to Use This Book}

\paragraph{For the General Reader} Focus on Chapters 1--2 for historical context, Chapter 5 for descriptive insights, and Chapter 10 for compelling case studies. Technical sections are clearly marked and can be skipped without losing the narrative thread.

\paragraph{For the Data Scientist} Chapters 3--4 detail our data pipeline and feature engineering. Chapters 6--9 contain the core statistical models. Appendix D provides full model specifications and diagnostic tests.

\paragraph{For the Sumo Enthusiast} Chapter 5's landscape analysis and Chapter 10's case studies will be of particular interest. Appendix B contains a comprehensive kimarite catalog with historical notes.

\paragraph{For Researchers} All data and code are available for replication and extension. We encourage independent verification of our findings and welcome corrections or improvements.

\section*{Acknowledgments}

This work would not have been possible without the meticulous record-keeping of the Japan Sumo Association and the dedication of the sumo data community, particularly the maintainers of Sumo-API and SumoDB.