\section{The Stable System}

The \japterm{heya}{heya} or stable system forms the backbone of professional sumo's organizational structure. These training centers function as:
\begin{itemize}
\item Residential facilities where wrestlers live, train, and eat together
\item Hierarchical social units with strict seniority systems
\item Economic entities that receive JSA subsidies and manage wrestler salaries
\item Cultural transmission mechanisms preserving technique and tradition
\end{itemize}

As of 2024, 44 stables operate under JSA oversight, down from a peak of 54 in the 1990s. Each stable maintains distinct training philosophies, recruitment networks, and technical specializations that measurably affect wrestler development and career outcomes.

Key structural rules governing stables include:
\begin{itemize}
\item Maximum of one foreign-born wrestler per stable
\item Stable-mates cannot face each other in regular tournament bouts
\item Mandatory retirement age of 65 for stablemasters
\item Inheritance and branching regulations for stable succession
\end{itemize}